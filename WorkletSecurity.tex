% CVS version control block - do not edit manually
% $RCSfile$
% $Revision$
% $Date$
% $Source$

% to compile this file into a pdf or ps use the following
% 1) latex WorkletSecurity.tex
% for ps: 2) dvips -K WorkletSecurity -o
% for pdf: 2) ps2pdf WorkletSecurity.ps WorkletSecurity.pdf

\documentclass[10pt]{article}
\usepackage{doublespace}
% \usepackage{geometry}
% \geometry{verbose,letterpaper,lmargin=45mm,rmargin=45mm}

\title{Worklet Security}
\author{Dan Phung (dp2041@cs.columbia.edu)}
\date{Draft of \today}

% \textheight=9.3in
% \topmargin=.2in
\renewcommand{\baselinestretch}{1.5}

\begin{document}

% \begin{singlespace}
\maketitle
% \end{singlespace}

\section{Abstract}

The goal of this project was to add security features to the Worklets
system while leaving previous functionality intact where it was
feasible.  I found possible communication related security risks to be
within the RMI transport layer, the WVM transporter layer and their
associated class loaders.  Intrusions are now minimized through the
addition of SSL (Secure Socket Layer) sockets to the RMI Registry, RMI
sockets, WVM Transporter and associated class loaders. and a
HostnameVerifier.  These security features require Java 1.4, which
includes JSSE (Java Secure Sockets Extension) and JCE (Java
Cryptography Extension).  With all previous functionally retained,
communication between WVMs can now be authenticated to ensure peer
validity and encrypted to uphold packet integrity.  These security
parameters can also be tailored according to the security needs at
each site.

\section{Intro}

The goal of the Worklet Security project was to analyze the Worklet
system to determine the security hazards and to implement provisions
for secure methods of communication that minimize those hazards.  The
Worklet system previously had no ability to authenticate the local or
remote WVM hosts, nor were the packets encrypted.  Without these
features there is the possibility for the transmission and execution of
malicious Worklets.

% SYMMETRIC and ASYMMETRIC keys.
\section{Background}

As stated in the JSSE documentation: \begin{quote}``Integrity means
that the data has not been modified or tampered with, and authenticity
means the data indeed comes from whoever claims to have created and
signed it.''\end{quote}

Therefore, to secure the Worklet system we must validate the local and
remote hosts and encrypt the Worklet code being sent.  The Worklet
security system is based off of public/private keys.  One of the
considerations in the implementation of a security architecture is
whether to use symmetric and asymmetric keys.

Symmetric algorithms involve the sharing of keys for encryption and
decryption.  A different set of keys must be kept for each pair of
users.  The weakness of these algorithms is that once the key is
intercepted the security of the system is no longer viable.  The
advantage of these algorithms is that they do not consume too much
computing power and are faster than asymmetric algorithms.  Typical
symmetric algorithm are DES (Data Encryption Standard), 3DES, and
BLOWFISH.

Asymmetric algorithms use pairs of public/private keys. The public key
is used for encryption and the private key, or secret key, is used for
decryption.  Usually the public key is shared to anyone that wants to
encrypt data, and only the receiver keeps the private key.  Because
the private key is kept secret the risk of the system is less than
with symmetric keys.  A typical set of keys for a single user is thus
a private key to decrypt incoming data and a set of public keys that
is used to send data to peers.  Public keys are also usually digitally
signed by third party certificate authorities (but they can be
self-signed) to ensure their validity upon distribution.  The strength
of this algorithm is the higher reliability in key integrity.  The
disadvantages are that these algorithms are more complex to handle and
usually take higher computing power and are thus much slower.  Typical
asymmetric algorithms are RSA (Rivest, Shamir, and Adelman) and DSA
(Digital Signature Algorithm).

The JSSE implementation used in the Worklet system is a combination of
symmetric and asymmetric
algorithms\footnote{http://java.sun.com/j2se/1.4/docs/guide/security/jsse/JSSERefGuide.html\#HowSSLWorks}.
It uses asymmetric key algorithms in the authentication phase of
communication and symmetric keys to encrypt the data in the transport
phase.  The use of both algorithms allows the advantages of both
algorithms can be used.

% this section to be the distributed Worklet Security documentation.
\section{Program Documentation}
The Worklet security system provides methods for securing Worklet
transmission.  The vulnerable objects are the RMI (Remote Method
Invocation) transport layer, the WVM (Worklet Virtual Machine)
transport layer and the class loaders.  Other objects that have or
provide methods for security measures are HostnameVerifier and
WorkletJunctions.

The RMI Transport layer involves an RMI registry, associated RMI
servers and class loaders.  Only one registry can be created on a
specific host:port to which many RMI servers can be bound.  A secure
registry authenticates all registry calls such as server lookups,
binds, and unbinds.  A secure RMI server uses custom secure sockets to
authenticate and encrypt all communication.  In the instance of a
secure server, the default RMI class loader is set to our own
implementation that uses a class loader with secure sockets.  Both
plain and secure RMI servers can be bound to plain registries but only
secure RMI servers can be bound to secure registries.  Communication
between RMI servers can only proceed plain-plain or secure-secure.
Depending on the security level (described in the Usage section) the
RMI server could have either only plain, only secure, or both class
loaders available to load classes.

The WVM transport layer includes server sockets and class loaders.
Depending on the level of security the transport layer can be created
with plain, secure, or both types of server sockets and associated
class loaders.  

A HostnameVerifier is also implemented and is used on the remote
(receiving) end.  This object is used by the remote host if the
hostname of the sending WVM is not recognized from the CA certificate
file.  The specification of the hosts that are allowed and denied is
described in the usage section.

There are situations where the user of the WorkletJunctions could be
uncertain of the security specifications at a certain WVM.  The user
can now specify parameters that allow the users to specify which
methods of transport (plain RMI, secure RMI, plain socket, secure
socket) the WorkletJunction should try and in what order.  The RMI
related methods specify what type of communication to try when looking
up the recipient RMI server, not the actual RMI server.  The socket
related features specify the type of socket to try connecting on.  The
ability to specify transport methods contributes to the robustness of
WorkletJunction communication.

\subsection{Usage}

Note that all previous functionality is retained.  This section
documents the features pertinent to Worklet Security.  I make a
delineation between the remote and local hosts.  The remote 
host represents the intended recipient of the Worklet while the
local host is the entity that is sending the Worklet.

\subsubsection{Remote Host}
To create a plain host you need not specify any parameters, like
this: 

\begin{verbatim}
java psl.worklets.WVM
\end{verbatim}

The name and port of the RMI server will default to WVM\_Host and
9100.  If a remote host is created in this manner communication will
not be authenticated nor encrypted.  You can specify the RMI server
name and port by using the switches -name and -port.  See java
psl.worklets.WVM -help for option details.

To create secure hosts you must specify at least the keys file,
password, and WVM properties file (WVM file).  Because the WVM file
must contain the keysfile and password it is usually sufficient to
simply specify the WVM file.  The keys file is the file containing the
public/private keys of the host and the password is the passphrase
specified in the creation of the keysfile.  For more information on
how to create a keysfile see the FAQ below.  The WVM file holds the
security specifications.  These security specifications can be
specified by setting environment variables with the -D java switch or
on the command line.  The parameter loading precedence is:
\begin{enumerate}
\item Environment variables
\item WVM file
\item command line specification
\end{enumerate}

The WVM file contains property=value pairs separated by white space.
The important parameters in the WVM file are the keysfile and
password as these are used by the Worklet system at different times.
There is also an internally needed parameter that needs to be set in
the WVM file, namely:
\begin{verbatim}
java.rmi.server.RMIClassLoaderSpi=psl.worklets.WVM_RMIClassLoaderSpi
\end{verbatim}

It is \textbf{very} important that the security of this file is set at
the operating systems level, meaning that the permissions on this file
should be set accordingly.  For example, for maximum security I would
set the permission to user read-only.  This can be done under Unix
with the following:

\begin{verbatim}
chmod a-rwx wvm_properties
chmod u+r wvm_properties
\end{verbatim}

Included with the distribution should be an example WVM file named
``wvm\_properties''.  It should look like this: 

\begin{verbatim}
# this first line is needed for internal purposes.
java.rmi.server.RMIClassLoaderSpi=psl.worklets.WVM_RMIClassLoaderSpi

# WVM host related parameters
WVM_RMI_PORT=9100	# port that the rmi registry will be created on
WVM_RMI_NAME=target	# name that the RMI server will be bound to

# Security related parameters 
WVM_KEYSFILE=testkeys	# file holding the public/private keys
WVM_PASSWORD=passphrase	# password into the keysfile. 
WVM_SSLCONTEXT=TLS	# SSL context instance to use
WVM_KEYMANAGER=SunX509	# key manager implementation type
WVM_KEYSTORE=JKS	# key store implementation type
WVM_RNG=SHA1PRNG	# random number generator algorithm to use
javax.net.ssl.trustStore=samplecacerts # location of the certified security certificates.
WVM_HOSTS_ALLOW=localhost,127.0.0.1 # allowed hosts
WVM_HOSTS_DENY=			    # denied hosts
WVM_FILE=wvm_properties	# reference to this file

# security level of the WVM.  
# 0: no security. 
# 1: low security = plain RMI Registry, secure RMI server, 
#     plain and secure server sockets, and 
#     plain and secure class loaders.
# 2: medium security = same as 1 but with a secure RMI Registry
# 3: high security = secure RMI Registry, secure RMI server, 
#    secure server socket and secure class loaders.
WVM_SECURITY_LEVEL=1	
\end{verbatim}

Notice the last parameter of WVM\_SECURITY\_LEVEL.  This parameter is
what you would modify to tailor the security level according to the
needs at a certain WVM.  A security level of 1 is the most flexible as
it creates a registry that will allow binding by both plain and secure
RMI servers.  Also created with a security level of 1 are plain and
secure server sockets and class loaders.  The only difference between
security level 1 and 2 is that the registry is secure in the latter.
A security level of 3 creates a WVM host that has a secure RMI server
that must be bound to a secure registry, a secure server socket, and only
secure class loaders.  There is a plain server socket created, but
that is used only for internal purposes for robust registry upkeep.

The WVM\_HOSTS\_ALLOW and WVM\_HOSTS\_DENY specify which hosts on the
remote end are allowed and denied.  If no hosts are specified in both
settings, then all hosts will be allowed.  These entries are the only
parameters that are concatenated, meaning that if you specify these
parameters in the system environment, the WVM file, and on the command
line then all entries will be used.

\subsubsection{Examples}


\begin{verbatim}
java psl.worklets.WVM -name target
\end{verbatim}
Create a plain host with the RMI name target on port 9100.

\begin{verbatim}
java psl.worklets.WVM -wvmfile wvm_properties
\end{verbatim}
Create a host according to the parameters in the WVM file.

\begin{verbatim}
java -DWVM_FILE=wvm_properties psl.worklets.WVM -name target -S 2
\end{verbatim}
Create a host according to the parameters in the WVM file, but with
the RMI name target and a security level of 2.

\subsubsection{Local Host}

The sending of secure Worklets involves the creation of a local WVM
host that bootstraps the Worklet into the WVM system.  Creation of
secure Worklets and WorkletJunctions is the same as described in the
Worklet documentation except that you need to set the WVM\_FILE system
property to your WVM file (described above).  This can either be done
with the -D java switch or within the application with the
System.setProperty method.  Included in the distribution should be an
example program ``SendSecure.java''.

Here is the portion within the example program that sets the security
parameters.  Also shown is the creation of the WVM.

\begin{verbatim}
psl.worklets.OptionsParser op = new psl.worklets.OptionsParser();
op.loadWVMFile(System.getProperty("WVM_FILE"));
op.setWVMProperties();

wvm = new psl.worklets.WVM(new Object(), InetAddress.getLocalHost().getHostAddress(), 
                           "SendSecure", op.port, op.keysfile, op.password,
                           op.ctx, op.kmf, op.ks, op.rng, op.securityLevel);
\end{verbatim}

The WVM\_FILE system property was set on the command line like this: \\
\begin{verbatim}
java -DWVM_FILE=wvm_properties SendSecure localhost target 9101 Apps.Face 1 0 mysend
\end{verbatim}

You can also manage the WorkletJunction transport methods by setting
the transportMethods to a combination of plainRMI, secureRMI,
plainSocket, and secureSocket.  The plainRMI and secureRMI keywords
specify the type of registry used by that server.  The type of server
cannot be specified because the parameters for which type of socket to
use is set at creation time.  The plainSocket and secureSocket
keywords specify which type of socket to communicate with.  The
security will always default to the ``parent'' if not set.  So if the
Worklet.isSecure and the WorkletJunction.isSecure have not been
specified, then the WorkletJunction will default to the security of
the current WVM system.  The default for plain systems is plainRMI and
plainSocket and the default for secure systems is secureRMI and
secureSocket.  If WorkletJunction.isSecure is set and the
WorkletJunction.isSecure has not been set then those WorkletJunctions
will default to the security level of the Worklet.  The last, and
highest priority level is at the WorkletJunction.  You can either use
the isSecure method or specify the methods through the
transportMethods function.  Using the transportMethods function alone
is sufficient, and will override the isSecure method.  Here's an
example of how to set the methods:

\begin{verbatim}
String[] tm = {"secureRMI", "plainRMI", "secureSocket", "plainSocket"};
wjxn.setTransportMethods(tm);
\end{verbatim}

\section{Program Documentation - Internal}
% need to talk about registry blah through sockets.
This section describes added and modified features in the Worklet
system.  These features include an SSL socket factory, robust RMI
registry handling, an implementation of RMIClassLoaderSpi and
HostnameVerifier, and an OptionsParser.

The SSL socket factory, WVM\_SSLSocketFactory, implemented in
the Worklet system was derived from the JSSE sample included in the
J2SDK 1.4 distribution.  I packaged the functionality of the JSSE
sample code together in one class and gave the user access to modify
the security parameters.  I also have WVM\_SSLSocketFactory
extending RMISocketFactory and implementing RMIServerSocketFactory,
RMIClientSocketFactory, and Serializable so that this factory can also
be used as our custom RMI socket factory.  The available methods are:
createSocket(), createServerSocket(), getSSLSocketFactory(), and
getSSLServerSocketFactory().  The reason why I have a private
initFactories() method is because when the object serialized, all the
members are instantiated as null, so they must be reinstantiated on the
remote side.  

For robust registry maintenance I modified the
WVM\_RMI\_Transporter.shutdown() method and added a
WVM\_Registry and RTU\_Registrar.  During WVM
instantiation a plain socket will always be created.  This socket will
be used by the RMI registry and server to broadcast registry creation
and rebinding requests during shutdown, along with the normal WVM
operations.  If the security level is 3, or high security, then the
plain socket will only accept RMI registry and server related
requests.  This implementation allows plain and secure RMI servers to
bind onto a registry and communicate events related to the registry.
The hostname, socket port number, and RMI name along with a randomly
created key comprise the registration information needed by our
registry to keep track of the bound servers.  I subclassed the default
RMI Registry in WVM\_Registry to add the features of
maintaining bound servers and restricting the rebinding method.  The
RTU\_RegistrarImpl, an implementation of the
RTU\_Registrar, acts as an intermediate between the registry
and the RTU (our RMI server.)  It creates and manages the registration
information for each server and negotiates the binding of the server
to our registry.

% need to talk about hostname verifier
A WVM\_HostnameVerifier was also added to the Worklets system as a
security extension.  This class was needed internally to verify hosts
not included in the CA certs file.  This object uses the
WVM\_HOSTS\_ALLOW and WVM\_HOSTS\_DENY system properties.  See the
``Program Documentation '' section for details on how to
set these.

% WVM_RMIClassLoaderSpi
The WVM\_RMIClassLoaderSpi allows us to specify the use of our own
class loader.  The default RMI class loader and system class loader
does not use secure sockets, nor are there ways to specify the use of
our sockets or socket factories.  The WVM\_RMIClassLoaderSpi specifies
our WVM\_ClassLoader which can be instructed to use secure sockets.
The specification to the system to use our RMI class loader is done by
setting the system environment variable as such:
\begin{verbatim}
java.rmi.server.RMIClassLoaderSpi=psl.worklets.WVM_RMIClassLoaderSpi.
\end{verbatim}
This parameter is usually put in the WVM file.  To ease the load of
our class loader and to maintain the speed of the Worklets system, the
WVM\_RMIClassLoaderSpi does not load the following classes: java.*,
javax.*, sun.*, sunw.*, and psl.*

I added an OptionsParser to ease the handling of the multitude
of added parameters that can be specified.  The OptionsParser
parses the command line as well as related environment variables and
the WVM file.  An OptionsParser has public access and can be
used to load in the security parameters in user programs.  For example
of this procedure see the ``Local side'' section in the ``Program
Documentation''.

\subsection{Possible errors}
\begin{itemize}

\item The most common errors pertain to the keysfile and the file
holding the CA certificates.  Either these files are spelled
incorrectly when specified or the password is incorrect.  Another
thing to check is that the permissions for the files at least give
user read rights.

\item RMI server cannot bind to registry. 

Upon trying to bind a RMI server, if you get this error: 
\begin{verbatim}
Shutting down; cannot bind to a non-local host: 128.59.23.10java.rmi.ServerException: RemoteException occurred in server thread; nested exception is: 
	java.rmi.UnmarshalException: Transport return code invalid
\end{verbatim}

Check that the security level of the server and registry are
compliant.  If it is secure, then it is expecting the registry to be
secure as well, and is thus communicating through secure protocols.
Security levels of 0 and 1 (none, and low) can bind on the same
registry.  Security levels of 2 and 3 (medium and high) can also bind
together on the same registry, but cannot be matched with levels 0 and
1.  Another issue to be aware of is the RMI class loader of the
Registry.  If the Registry is created with a secure RMI server, the
instatiation of the server sets the RMI class loader to our secure
implementation and essentially sets the RMI class loader for that JVM.
Therefore when other non-secure servers try to bind, the registry
interacts with the secure RMI class loader, and a remote exception is
thrown.  To prevent this from happening you should create the registry
with a plain server.

\end{itemize}

\subsection{Frequently Asked Questions (FAQ)}
\begin{itemize}

\item How do I create my own public/private keystore and how
do I self certify it?

Here is an example of how to create a keystore named testkeys with the
password asd123.  In general you do not want to specify the password
in the command used.  When not specified on the command line the
keytool program will prompt you for a password.  The following 
command line should all be on the same line.

keytool -genkey -dname "cn=Foo Bar, ou=Columbia University, o=PSL, \
        c=US" -alias foobar -keypass asd123 -keystore testkeys \
        -storepass asd123 -validity 180

The example also self-certifies the keystore with a validity time of
180 days.  A better way to certify the keystore would be to create a
certificate request and then send it to a third party certifier (see
Java JSSE docs).  The entries that reside in the WVM file that are
related to the file that was just created are:
\begin{verbatim}
WVM_KEYSFILE=testkeys
WVM_PASSWORD=asd123
javax.net.ssl.trustStore=testkeys
\end{verbatim}

to view details about the keystore you can do:
\begin{verbatim}
keytool -list -keystore testkeys
\end{verbatim}

For for information see:  \\
http://java.sun.com/j2se/1.4/docs/guide/security/SecurityToolsSummary.html

\item Where do I find out more information about the available
algorithms that I can use? 

See the Java documentation about security.  Specifically, look at the
JSSE and JCE docs.

\item What about java policies? 

In my implementation the java policies do not play a role.

\end{itemize}

% DONE
\section{Conclusion}

The security features implemented for the Worklet Security project
provides the methods for secure Worklet communication.  Security
measurements can only at best minimize the risks inherent in network
communication and is only as effective as the awareness of the user.

\section{Future Enhancements}
\begin{itemize}
\item have a better way to pass in the password to the WVM\_RMIClassLoaderSpi.
\end{itemize}

\end{document}

DEMO
overview the Worklets system.  show that the holes are at:
RMI: registry-RMI server, RMI server-RMI server, RMI Class Loader
sockets: socket-socket, ClassLoader.

First we should talk about the different transports used in the
Worklet system.

RMI: 
- true -, 
- true +/no class loading
- true +/with class loading

sockets: 
- true -, 
- true +/no class loading
- true +/with class loading

Registry
- creation of plain registry > bind to registry
- creation of secure registry > bind to plain and secure registry
- RMI send to plain
- RMI send to secure

power point presentation
before security
- danger to WVM
- danger to worklet
- no RMI authentication
- no socket authetication
- anyone can see the code being sent between sockets

demo: 

1) <RMI REGISTRIES>

we should show that plain communication still works bring up a plain
external registry bind both a secure and plain server to those.
explain that these are both communicating with plain communication
through Naming.

create a secure external registry.  show how only secure 
WVM's can bind to it.

explain the plain server contacting secure server problem, and
how if security is downgraded, it will be lost in that registry
system.

it was possible to kill a RMI host, binded to a certain name, then to
bind another host to that name.  now the servers are authenticated
through the registry before binding can occur (or even a lookup).
also, only psl.worklets.WVM_RMI_Transporters can rebind.

<WVM>

security levels.  level 1 is everything is secure except for the
registry.  the registry is still plain to allow binding with other
non-secure RMI servers, also, plain sockets are still up.  level 2 is
all secure, with the secure registry.  also, plain sockets are still
up. level 3 is secure only. only secure rmi and sockets are available, 
along with only secure codebases.

<CLASSLOADING, CODEBASE>
if security is set, then we have a https class server, but we
also still the normal http server.  only if the security
level is at 3, secure only, is only the https server up.

<WVMFILE>
explain all the fields
show java psl.worklets.WVM --help
explain the scaffolding of the parameters, environment, wvmfile, 
command line.

<RMIClassLoaderSpi>
 --needs the wvmfile.  security on the file must be at the OS level, 
setting of file permissions.  ask for other suggestions on how do this.

<HOSTNAME VERIFIER>
the HostnameVerifier is for the remote host to allow class loading.

<WORLETJUNCTION>
ability to set the transport methods.  

so there are ways to go between secure and plain WVM's

RMI -/-> sRMI

host-remote
WVM-WVM: only plain transports are available.
WVM-sWVM: cannot send through RMI
depends: securityLevel 1,2: can send through plain socket. 
         if securityLevel 3, then cannot send through sockets.

sWVM    WVM: 

sWVM    sWVM: depends
RMI, 

so security is not at the workletJunction level.  it is between hosts.

<KEYSFILE, CA CERTS's>

So we are going to use symmteric keys.  The creation and management of
these keys and certificates are discussed in detail with the JSSE
documentation that is a part of Java SDK 1.4.  The tool used to create
and manage the keystore is ``keytool.''  Here I will briefly go
through an example of key and certificate creation.  The keystore and
its password are the required parameters for Worklet Security that
need to reside in the WVM file.

Here is an example of how to create a keystore named testkeys
with the password asd123.  In general you do not want to specify
the password in the command used.  Rather, when not specified 
on the command line the keytool program will prompt you for a
password.

keytool -genkey -dname "cn=Foo Bar, ou=PSL, o=PSL, c=US" \
      -alias foobar -keypass asd123 -keystore testkeys \
      -storepass asd123 -validity 180

The example also self-certifies the keystore with a validity time of
180 days.  A better way to certify the keystore would be to create a
certificate request and then send it to a third party certifier (see
java jssse docs).  The entries that reside in the WVM file are that
are related to the file that was just created are:
WVM_KEYSFILE=testkeys
WVM_PASSWORD=asd123
javax.net.ssl.trustStore=/home/mrgray/src/psl/worklets/testkeys

to view details about the keystore you can do:
keytool -list -keystore testkeys



- The (WVM) has its own method of contacting peers to broadcast the RMI
registry shutdown.
